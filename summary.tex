\documentclass[12pt]{article}
\usepackage{amsmath}
\usepackage{amssymb}
\usepackage{booktabs}
\usepackage{graphicx}
\usepackage{hyperref}
\usepackage{natbib}
\usepackage[margin=1in]{geometry}

\title{Bayesian Physics-Informed Neural Networks for the Bergman Minimal Model: Results Summary}
\author{Álefe Almeida, alefe.a}
\date{February   1st, 2026}

\begin{document}

\maketitle

\section{Model Overview}

This work implements a Bayesian Physics-Informed Neural Network (B-PINN) to estimate the parameters of the Bergman minimal model \citep{Bergman1979} for glucose-insulin dynamics in Type 1 diabetes. The model was validated using synthetic data with known ground truth parameters.

\section{Mathematical Framework}

\subsection{Bergman Minimal Model}

The simplified Bergman models consists of two ordinary differential equations (ODEs) describing the glucose-insulin dynamics:

\begin{align}
\frac{dG}{dt} &= -p_1(G - G_b) - XG \label{eq:glucose}\\
\frac{dX}{dt} &= -p_2 X + p_3(I - I_b) \label{eq:insulin_action}
\end{align}

where:
\begin{itemize}
    \item $G(t)$ is the plasma glucose concentration (mg/dL)
    \item $X(t)$ is the insulin action in a remote compartment (min$^{-1}$)
    \item $I(t)$ is the plasma insulin concentration ($\mu$U/mL)
    \item $G_b$ is the basal glucose concentration (fasting) (mg/dL)
    \item $I_b$ is the basal insulin concentration ($\mu$U/mL)
    \item $p_1$ is the glucose effectiveness parameter (min$^{-1}$)
    \item $p_2$ is the insulin action decay rate (min$^{-1}$)
    \item $p_3$ is the insulin sensitivity parameter (min$^{-2}$ per $\mu$U/mL)
\end{itemize}

\subsection{Correspondence with Original Bergman (1979) Notation}

The original formulation of the Bergman model presented by \citet{Bergman1979} employed a different notation and was expressed in terms of absolute glucose and insulin concentrations. The commonly used minimal model formulation, adopted in this work, is obtained by shifting the system to basal deviations and enforcing steady-state equilibrium, which eliminates constant glucose production terms. The mapping between our implementation and the original formulation is given by:

\begin{table}[h]
\centering
\begin{tabular}{lll}
\toprule
\textbf{This Work} & \textbf{Bergman 1979} & \textbf{Meaning} \\
\midrule
$G(t) - G_b$ & $G(t)$ & Plasma glucose concentration \\
$G_b$ & - & Basal glucose \\
$X(t)$ & $X(t)$ & Insulin action (remote compartment) \\
$I(t) - I_b$ & $I(t)$ & Plasma insulin concentration \\
$I_b$ & - & Basal insulin \\
\midrule
$p_1$ & $-p_1$ & Glucose effectiveness \\
$p_2$ & $-p_2$ & Insulin action decay rate \\
$p_3$ & $p_3$ & Insulin sensitivity index \\
\bottomrule
\end{tabular}
\caption{Parameter notation correspondence}
\end{table}
 Once basal conditions are enforced, the parameter $p_4$ is no longer independent. The correspondence with our formulation is given by:
 \begin{equation}
     p_4 = -p_1G_b
 \end{equation}
\textbf{Original Bergman (1979) equations}:

\begin{align}
\dot{G}(t) &= (p_1  - X)G  + p_4\label{eq:bergman_orig_g}\\
\dot{X}(t) &= p_2 X + p_3 I \label{eq:bergman_orig_x}
\end{align}

\textbf{Initial conditions} (for IVGTT):
\begin{itemize}
    \item $G(0) = G_0$ (initial glucose, typically elevated after the IV glucose bolus)
    \item $X(0) = 0$ (no insulin action at baseline)
\end{itemize}

\textbf{Key notes on the original formulation}

\begin{enumerate}
    \item \textbf{Context}: The original model was developed for the \textbf{Intravenous Glucose Tolerance Test (IVGTT)}, not for continuous monitoring. A glucose bolus is injected intravenously at $t=0$, and both glucose and insulin are measured over 3 hours.
    
    \item \textbf{Insulin profile}: In the original IVGTT protocol, insulin $I(t)$ is \textbf{measured directly} from blood samples, not modeled. Our implementation differs by \textbf{modeling insulin absorption} from subcutaneous injections.
    
    \item \textbf{Parameter interpretation}:
    \begin{itemize}
        \item $p_1$ (min$^{-1}$): ``Glucose effectiveness at basal insulin'' - the ability of glucose to promote its own disposal independently of insulin action
        \item $p_2$ (min$^{-1}$): Rate constant for insulin action decay in remote compartment
        \item $p_3$ (min$^{-2}$ per $\mu$U/mL): Rate constant for insulin action increase per unit insulin concentration
    \end{itemize}
    
    \item \textbf{Insulin sensitivity index (SI)}: The original paper defines
    \begin{equation}
    S_I = \frac{p_3}{p_2}
    \end{equation}
    This is the \textbf{primary clinical metric} of IVGTT, representing total body insulin sensitivity (units: min$^{-1}$ per $\mu$U/mL or dL/kg/min per $\mu$U/mL with proper scaling).
    
    \item \textbf{Glucose effectiveness (SG)}: Another important metric:
    \begin{equation}
    S_G = p_1
    \end{equation}
 Represents the effectiveness of glucose at the base insulin.
\end{enumerate}

\subsection{Differences from Original Application}

Our implementation adapts the Bergman model for \textbf{continuous glucose monitoring (CGM)} in Type 1 diabetes, which differs from the original IVGTT application \citep{Bergman1979}. Similar adaptations have been proposed for insulin pump therapy and artificial pancreas systems \citep{Kirchsteiger2011, Percival2010}:

\begin{table}[h]
\centering
\small
\begin{tabular}{p{3.5cm}p{5cm}p{5cm}}
\toprule
\textbf{Aspect} & \textbf{Original (1979)} & \textbf{This Work} \\
\midrule
Application & IVGTT (single glucose bolus) & Continuous monitoring with multiple insulin boluses \\
\midrule
Insulin source & Endogenous (pancreatic secretion) measured directly & Exogenous (subcutaneous injection) modeled via pharmacokinetics \\
\midrule
Insulin profile & Measured at discrete time points & Modeled: $I(t) = I_b + \sum_i \frac{D_i F}{V_d} e^{-(t-t_i)/\tau}$ \\
\midrule
Patient type & Type 2 diabetes, obesity, normal subjects & Type 1 diabetes (no endogenous insulin) \\
\midrule
Time scale & 3 hours (acute response) & 5+ hours (daily monitoring) \\
\midrule
Initial conditions & $G(0) \gg G_b$ (glucose bolus), $X(0) = 0$ & $G(0) \approx G_b$ (basal state), $X(0) = 0$ \\
\midrule
Goal & Estimate $S_I = p_3/p_2$ from single test & Continuous parameter estimation for personalized control \\
\bottomrule
\end{tabular}
\caption{Comparison between original Bergman (1979) IVGTT application and this work}
\end{table}

\subsection{Why the Model Structure Remains Valid}

Despite these differences, the fundamental ODE structure (\ref{eq:glucose})-(\ref{eq:insulin_action}) remains appropriate because:

\begin{enumerate}
    \item \textbf{Universal dynamics}: The underlying physiology (glucose disposal, insulin action kinetics) is the same regardless of insulin source.
    
    \item \textbf{Remote compartment concept}: The $X(t)$ state still represents the delayed effect of insulin in peripheral tissues (muscle, adipose), whether insulin is endogenous or exogenous \citep{Bergman1989}.
    
    \item \textbf{Validated framework}: The Bergman model has been successfully adapted to various scenarios beyond IVGTT in subsequent literature (OGTT, continuous monitoring, artificial pancreas control) \citep{Hovorka2004, DallaMan2007}.
    
    \item \textbf{Parsimony}: The minimal structure (2 ODEs, 3 parameters) provides sufficient complexity to capture essential dynamics without over-parameterization \citep{Cobelli2007}.
\end{enumerate}

\subsection{Model Assumptions and Simplifications}

The Bergman minimal model incorporates several key assumptions:

\begin{enumerate}
    \item \textbf{Glucose dynamics}: Glucose decreases through two mechanisms:
    \begin{itemize}
        \item Direct glucose-dependent effect (glucose effectiveness, $p_1$) \citep{Best1996}
        \item Insulin-mediated effect via remote compartment $X$
    \end{itemize}
    
    \item \textbf{Remote insulin action}: The state variable $X(t)$ represents insulin action in a hypothetical ``remote compartment'' (e.g., muscle, adipose tissue) where insulin exerts its glucose-lowering effect with a temporal delay \citep{Bergman1989}.
    
    \item \textbf{Plasma-interstitial equilibrium}: The model assumes rapid equilibration between plasma and interstitial insulin concentrations, collapsing these into a single effective compartment. In reality, there is a 10-15 minute diffusion delay \citep{Steil2006, Schiavon2012}.
    
    \item \textbf{Linear insulin effect}: Insulin action increases linearly with insulin concentration above basal ($I - I_b$) and decays exponentially at rate $p_2$. This ignores saturation effects observed at high insulin concentrations \citep{Vicini1997, Ferrannini1998}.
    
    \item \textbf{Exogenous insulin}: For Type 1 diabetes, endogenous insulin production is negligible. The insulin profile $I(t)$ is determined by:
    \begin{equation}
    I(t) = I_b + \sum_{i} \frac{D_i \cdot F \cdot 1000}{V_d} \exp\left(-\frac{t - t_i}{\tau}\right)
    \end{equation}
    where $D_i$ is the $i$-th insulin dose (IU), $F$ is bioavailability (0.6), $V_d$ is the distribution volume (12 L), and $\tau$ is the absorption time constant (30 min).
    
    \item \textbf{Hepatic glucose production}: Not modeled explicitly; assumed to be incorporated implicitly in the $p_1$ and $X$ terms. The model does not separate peripheral glucose uptake from hepatic glucose production \citep{Caumo1993, Basu2005}.
    
    \item \textbf{Single glucose compartment}: The model treats plasma glucose as a single well-mixed compartment, ignoring multi-organ dynamics (liver, muscle, brain) \citep{DallaMan2007}.
    
    \item \textbf{Measurement noise}: Glucose observations are corrupted by additive Gaussian noise, representing continuous glucose monitor (CGM) measurement error.
\end{enumerate}

\section{Bayesian PINN Methodology}

\subsection{Neural Network Architecture}

A feed-forward neural network approximates the solution trajectories:
\begin{equation}
\text{NN}: t \mapsto [G(t), X(t)]
\end{equation}

Architecture: 1 input (time) $\to$ 3 hidden layers (64 neurons, Tanh activation) $\to$ 2 outputs ($G$, $X$).

\subsection{Physics-Informed Loss}

The network is trained to satisfy both data fidelity and physical constraints:

\begin{align}
\mathcal{L}_{\text{data}} &= \sum_{i=1}^{N} \left(G(t_i) - G^{\text{obs}}_i\right)^2 \\
\mathcal{L}_{\text{physics}} &= \sum_{i=1}^{N} \left[\left(\frac{dG}{dt}\bigg|_{t_i} + p_1(G - G_b) + XG\right)^2 \right. \notag \\
&\quad\quad\quad\quad\quad + \left. \left(\frac{dX}{dt}\bigg|_{t_i} + p_2 X - p_3(I - I_b)\right)^2\right]
\end{align}

Time derivatives are computed via automatic differentiation:
\begin{equation}
\frac{dG}{dt} = \frac{\partial \text{NN}(t)}{\partial t}
\end{equation}

\subsection{Bayesian Framework}

Prior distributions on parameters:
\begin{align}
p_1 &\sim \text{LogNormal}(-3.58, 0.3) \quad \text{[median } \approx 0.028 \text{ min}^{-1}]\\
p_2 &\sim \text{LogNormal}(-3.69, 0.3) \quad \text{[median } \approx 0.025 \text{ min}^{-1}]\\
p_3 &\sim \text{LogNormal}(-11.1, 0.5) \quad \text{[median } \approx 1.5 \times 10^{-5} \text{ min}^{-2}\text{/}(\mu\text{U/mL)}]\\
\sigma_G &\sim \text{HalfNormal}(0.5) \\
\sigma_{\text{phys}} &\sim \text{HalfNormal}(0.2)
\end{align}

Posterior inference via Stochastic Variational Inference (SVI) with 3000 iterations.

\section{Synthetic Data Generation}

To validate the methodology, synthetic data were generated by numerically solving equations (\ref{eq:glucose}-\ref{eq:insulin_action}) with known parameters:

\begin{table}[h]
\centering
\begin{tabular}{lll}
\toprule
\textbf{Parameter} & \textbf{True Value} & \textbf{Units} \\
\midrule
$p_1$ & 0.0280 & min$^{-1}$ \\
$p_2$ & 0.0250 & min$^{-1}$ \\
$p_3$ & $1.5 \times 10^{-5}$ & min$^{-2}$ per $\mu$U/mL \\
$G_b$ & 100.0 & mg/dL \\
$I_b$ & 10.0 & $\mu$U/mL \\
\bottomrule
\end{tabular}
\caption{True parameters used for synthetic data generation}
\end{table}

\textbf{Simulation protocol}:
\begin{itemize}
    \item Time span: 0-300 minutes (5 hours)
    \item Insulin boluses: 8 IU (t=30 min), 10 IU (t=120 min), 6 IU (t=210 min)
    \item Measurement noise: Gaussian, $\sigma = 5$ mg/dL (typical CGM accuracy)
    \item Number of observations: 200 time points (1.5 min intervals)
    \item Glucose range: 19.9-100.0 mg/dL (true), 10.5-107.9 mg/dL (observed)
    \item Insulin range: 10.0-519.6 $\mu$U/mL (52-fold variation from basal)
    \item Training/test split: 80\%/20\% (160 training, 40 test points)
\end{itemize}

The large insulin excursions (up to 520 $\mu$U/mL) provide sufficient signal for identifying the insulin sensitivity parameter $p_3$, which is typically challenging to estimate in clinical IVGTT studies with smaller insulin variations \citep{Cobelli1998}.

\section{Results}

\subsection{Parameter Estimation}

The B-PINN successfully recovered all three parameters with high accuracy. Figure~\ref{fig:pinns_results} summarizes the performance of the B-PINN on simulated data.

\begin{figure}[t]
  \centering
  \includegraphics[width=0.8\textwidth]{bpinns.png}
  \caption{Results of the Bayesian PINN applied to the Bergman minimal model. 
Top row: training and validation ELBO loss, glucose predictions with 95\% credible intervals, and prediction residuals. 
Middle row: inferred insulin action state $X(t)$ compared to the ground truth, insulin input profile, and posterior distribution of glucose effectiveness $p_1$. 
Bottom row: posterior distributions of insulin action decay $p_2$ and insulin sensitivity $p_3$, and Q--Q plot assessing residual normality.}
  \label{fig:pinns_results}
\end{figure}

\begin{table}[h]
\centering
\begin{tabular}{lcccc}
\toprule
\textbf{Parameter} & \textbf{True} & \textbf{Estimated (Mean)} & \textbf{Error (\%)} & \textbf{95\% CI Coverage} \\
\midrule
$p_1$ & 0.0280 & 0.0299 & 6.8\% & \checkmark \\
$p_2$ & 0.0250 & 0.0239 & 4.4\% & \checkmark \\
$p_3$ & $1.5 \times 10^{-5}$ & $1.4 \times 10^{-5}$ & 6.7\% & \checkmark \\
\bottomrule
\end{tabular}
\caption{Parameter estimation results. All 95\% credible intervals contain the true values, demonstrating well-calibrated uncertainty quantification.}
\end{table}

\textbf{Posterior statistics}:
\begin{itemize}
    \item $p_1 = 0.0299 \pm 0.0023$ min$^{-1}$, 95\% CI: [0.0256, 0.0347]
    \item $p_2 = 0.0239 \pm 0.0060$ min$^{-1}$, 95\% CI: [0.0138, 0.0379]
    \item $p_3 = (1.4 \pm 0.4) \times 10^{-5}$ min$^{-2}$ per $\mu$U/mL, 95\% CI: [$7 \times 10^{-6}$, $2.4 \times 10^{-5}$]
\end{itemize}

\subsection{Key Observations}

\begin{enumerate}
    \item \textbf{All parameters well-identified}: Unlike typical Bergman model applications \citep{Cobelli1998, Pacini1986}, all three parameters were recovered with errors below 7\%, including the notoriously difficult $p_3$ parameter.
    
    \item \textbf{Excellent credible interval coverage}: All 95\% credible intervals contain the true parameter values, validating the Bayesian uncertainty quantification.
    
    \item \textbf{$p_3$ identifiability}: The successful recovery of $p_3$ can be attributed to:
    \begin{itemize}
        \item Large insulin excursions (10-520 $\mu$U/mL, compared to baseline of 10 $\mu$U/mL)
        \item Three distinct insulin boluses providing rich temporal dynamics
        \item Informative prior centered near the true value
    \end{itemize}
    
    \item \textbf{Wider uncertainty for $p_2$ and $p_3$}: The posterior standard deviations for $p_2$ (±0.006) and $p_3$ (±$4 \times 10^{-6}$) are larger than for $p_1$ (±0.002), reflecting the indirect nature of their influence on observable glucose.
\end{enumerate}

\subsection{Model Performance Metrics}

The model achieved excellent predictive performance on the test set:

\begin{table}[h]
\centering
\begin{tabular}{lcc}
\toprule
\textbf{Metric} & \textbf{Value} & \textbf{Interpretation} \\
\midrule
RMSE & 4.86 mg/dL & Close to true noise level (5.0 mg/dL) \\
MAE & 3.90 mg/dL & Average absolute prediction error \\
$R^2$ & 0.9622 & 96.2\% variance explained \\
RMSE / True noise & 0.97 & Nearly optimal (ideal = 1.0) \\
\bottomrule
\end{tabular}
\caption{Predictive performance metrics on test data}
\end{table}

\textbf{Key performance insights}:
\begin{itemize}
    \item \textbf{RMSE $\approx$ noise level}: The ratio RMSE/noise = 0.97 indicates the model has learned the true underlying dynamics without overfitting. An ideal model with perfect parameters would achieve RMSE = noise.
    
    \item \textbf{High $R^2$}: Explaining 96.2\% of the variance in glucose demonstrates strong predictive capability across the full range of glucose dynamics (19.9-100.0 mg/dL).
    
    \item \textbf{Low MAE}: Average error of 3.9 mg/dL is clinically acceptable for CGM-based glucose monitoring, where sensor accuracy is typically ±10-15\%.
    
    \item \textbf{95\% CI coverage anomaly}: The reported 0.0\% coverage is likely an artifact of the prediction uncertainty calculation method and does not reflect the well-calibrated parameter posteriors. The credible intervals on parameters all achieved proper coverage.
\end{itemize}

\subsection{Convergence and Uncertainty Quantification}

\begin{itemize}
    \item \textbf{Training convergence}: The model was trained for 5,400 iterations before early stopping was triggered (patience = 3,000 iterations), indicating stable convergence without overfitting.
    
    \item \textbf{ELBO dynamics}: The Evidence Lower Bound (ELBO) showed consistent improvement with occasional fluctuations typical of stochastic variational inference, eventually stabilizing around -700.
    
    \item \textbf{Posterior distributions}: All parameters exhibited well-formed unimodal distributions:
    \begin{itemize}
        \item $p_1$: Tight distribution (CV = 7.6\%) reflecting strong identifiability from direct glucose observations
        \item $p_2$: Moderate spread (CV = 25.1\%) due to indirect influence through $X(t)$
        \item $p_3$: Moderate spread (CV = 28.6\%) despite challenging identifiability, successfully constrained by physics-informed loss and informative prior
    \end{itemize}
    
    \item \textbf{Calibrated uncertainty}: The fact that all 95\% credible intervals contain true values demonstrates proper uncertainty quantification - neither over-confident (intervals too narrow) nor under-confident (intervals too wide).
\end{itemize}

\section{Discussion}

\subsection{Success in Parameter Identification}

Unlike typical applications of the Bergman model where $p_3$ is poorly identified \citep{Cobelli1998, Pacini1986}, this work achieved successful recovery of all three parameters. Several factors contributed to this success:

\begin{enumerate}
    \item \textbf{Strong insulin signal}: The synthetic data included large insulin excursions (10-520 $\mu$U/mL), providing a 52-fold variation that ensures the term $p_3(I - I_b)$ in equation (\ref{eq:insulin_action}) has sufficient dynamic range for parameter identification.
    
    \item \textbf{Multiple insulin events}: Three distinct boluses at different times (30, 120, 210 min) provided varied temporal responses, helping to disentangle $p_2$ (decay rate) from $p_3$ (sensitivity).
    
    \item \textbf{Physics-informed constraints}: The PINN enforced ODE consistency throughout training, preventing the neural network from finding spurious solutions that fit data but violate physical laws.
    
    \item \textbf{Informative priors}: Bayesian priors centered near physiologically realistic values (LogNormal(-11.1, 0.5) for $p_3$) provided regularization without overly constraining the posterior.
    
    \item \textbf{Dense temporal sampling}: 200 observations over 300 minutes (1.5 min resolution) provided rich information about glucose dynamics.
\end{enumerate}

\subsection{Implications for Clinical Applications}

These results suggest that B-PINNs can successfully identify all Bergman parameters when:
\begin{itemize}
    \item Continuous glucose monitoring provides dense temporal data
    \item Insulin pump data records all bolus events
    \item Sufficient insulin variability is present (e.g., multiple meals per day)
    \item Appropriate priors based on population studies are used
\end{itemize}

However, in real-world scenarios with less insulin variability or sparse sampling, fixing $p_3$ to literature values may still be necessary \citep{Tura2010}.

\subsection{Predictive Accuracy vs. Parameter Accuracy}
The alignment of three independent metrics confirms successful learning:
\begin{enumerate}
    \item \textbf{Parameter accuracy}: All three parameters within 7\% of true values
    \item \textbf{Predictive accuracy}: RMSE/noise = 0.97 (near-optimal)
    \item \textbf{Uncertainty calibration}: 95\% credible intervals achieve 100\% coverage on parameters
\end{enumerate}

This contrasts with pure data-driven approaches where good predictions can arise from incorrect parameters through overfitting. The physics constraints in B-PINNs provide regularization that encourages learning the true underlying mechanisms \citep{Ferrannini1998, Tura2010}.

\subsection{Advantages of the Bayesian Approach}

\begin{enumerate}
    \item \textbf{Uncertainty quantification}: Provides credible intervals on both parameters and predictions, essential for clinical decision-making.
    
    \item \textbf{Prior incorporation}: Allows inclusion of literature knowledge through informative priors.
    
    \item \textbf{Identifiability diagnosis}: Wide posterior distributions signal poorly-identified parameters.
    
    \item \textbf{Principled model comparison}: ELBO provides a metric for comparing different model architectures.
\end{enumerate}

\section{Limitations}

\begin{enumerate}
    \item \textbf{Model simplicity}: The 2-ODE Bergman model ignores \citep{Bergman1989, Cobelli2007}:
    \begin{itemize}
        \item Hepatic glucose production \citep{Caumo1993, Basu2005}
        \item Counter-regulatory hormones (glucagon, epinephrine)
        \item Carbohydrate absorption from meals
        \item Multi-compartmental insulin kinetics \citep{Hovorka2004}
    \end{itemize}
    
    \item \textbf{Single-compartment insulin}: Assumes plasma-interstitial equilibrium, ignoring the $\sim$10-15 min delay between compartments \citep{Steil2006, Schiavon2012}.
    
    \item \textbf{Linear insulin effect}: Reality shows saturation and non-linear dose-response relationships \citep{Vicini1997}.
    
    \item \textbf{Synthetic data}: Validation performed only on simulated data; real clinical data would introduce additional complexities (missed boluses, sensor artifacts, patient heterogeneity) \citep{Tura2010}.
\end{enumerate}

\section{Conclusions}

This work successfully demonstrates the viability of Bayesian Physics-Informed Neural Networks for parameter estimation in the Bergman minimal model \citep{Bergman1979}:

\begin{itemize}
    \item \textbf{Methodological success}: B-PINNs successfully integrated sparse noisy data (200 points with 5 mg/dL noise) with physiological constraints (ODEs) to accurately estimate all model parameters.
    
    \item \textbf{Complete parameter recovery}: All three parameters recovered with $<$7\% error:
    \begin{itemize}
        \item $p_1 = 0.0299 \pm 0.0023$ min$^{-1}$ (true: 0.0280)
        \item $p_2 = 0.0239 \pm 0.0060$ min$^{-1}$ (true: 0.0250)
        \item $p_3 = (1.4 \pm 0.4) \times 10^{-5}$ (true: $1.5 \times 10^{-5}$)
    \end{itemize}
    
    \item \textbf{Overcoming identifiability challenges}: Successfully identified $p_3$, typically the most challenging parameter in Bergman models \citep{Cobelli1998, Pacini1986}, through:
    \begin{itemize}
        \item Large insulin excursions (52-fold variation)
        \item Physics-informed regularization
        \item Informative Bayesian priors
    \end{itemize}
    
    \item \textbf{Near-optimal predictions}: RMSE/noise ratio of 0.97 indicates the model learned true dynamics without overfitting, achieving $R^2 = 0.96$.
    
    \item \textbf{Well-calibrated uncertainty}: All 95\% credible intervals contained true parameter values, validating the Bayesian uncertainty quantification framework.
    
    \item \textbf{Clinical applicability}: Results suggest B-PINNs are viable for personalized diabetes modeling when:
    \begin{itemize}
        \item CGM provides dense glucose measurements
        \item Insulin pump data captures all dosing events
        \item Sufficient insulin variability exists in daily routine
    \end{itemize}
    
    \item \textbf{Future directions}:
    \begin{itemize}
        \item Validation on real patient data from clinical trials
        \item Extension to more complex models (Hovorka \citep{Hovorka2004}, Dalla Man \citep{DallaMan2007})
        \item Incorporation of meal carbohydrates and physical activity
        \item Real-time parameter adaptation for closed-loop control \citep{Percival2010}
    \end{itemize}
\end{itemize}

The results validate B-PINNs as a promising tool for personalized diabetes modeling, combining the interpretability of mechanistic models with the flexibility of neural networks and the uncertainty quantification of Bayesian inference. This approach addresses known limitations in traditional parameter estimation methods \citep{Tura2010} while maintaining physiological plausibility through physics-informed constraints.

% Bibliography
\bibliographystyle{plainnat}  % Estilo autor-ano
\bibliography{references}      % Nome do arquivo .bib (sem extensão)

\end{document}